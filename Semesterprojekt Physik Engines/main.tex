\documentclass[11pt,paper=A4, margin=1cm]{article}
% Language setting
% Replace `english' with e.g. `spanish' to change the document language
\usepackage[german]{babel}
\usepackage{xcolor}
% Set page size and margins
% Replace `letterpaper' with `a4paper' for UK/EU standard size
\usepackage[letterpaper,top=2cm,bottom=2cm,left=3cm,right=3cm,marginparwidth=1.75cm]{geometry}% Useful packages
\usepackage{amsmath}
\usepackage{amsfonts} 
\usepackage{amssymb}
\usepackage{mathtools}
\usepackage{graphicx}
\usepackage[colorlinks=true, allcolors=blue]{hyperref}
\usepackage{float}
\usepackage{multicol}
\usepackage{xcolor}
\usepackage{amsfonts}
\usepackage[utf8]{inputenc}
\usepackage{pgfplots}
\usepackage{listings}
\usepackage[linewidth=1pt]{mdframed}
\usepackage{lipsum}
\usepackage{subfiles} % Best loaded last in the preamble
\usepackage[backend=biber]{biblatex}
\usepackage{csquotes}
% For labeling sub figures
\usepackage{subfigure}

\addbibresource{PE_Lab.bib}
\pgfplotsset{compat=1.18}

\title{Semesterprojekt Physik Engines}
\author{Kim Lan Vu, Michel Steiner, Asha Schwegler}




\begin{document}

\maketitle
\newpage

\tableofcontents



\newpage

\section {Zusammenfassung}
\subfile{subfiles/Zusammenfassung}

\section {Aufbau des Experiments}
\subfile{subfiles/AufbauDesExperiments}

\section {Physikalische Beschreibung der einzelnen Vorgänge}
\subfile{subfiles/PhysikalischeBeschreibung}

\section {Beschreibung der Implentierung inklusive Screenshots aus Unity}
\subfile{subfiles/Beschreibung der Implentierung inklusive Screenshots aus Unity}

\section {Rückblick und Lehren aus dem Versuch}
\subfile{subfiles/Rueckblick und Lehren aus dem Versuch}

\section {Resultate mit grafischer Darstellung}
\subfile{subfiles/ResultateMitGrafischerDarstellung.tex}

\appendix
\section {Anhang}
\subfile{subfiles/Anhang}

\newpage

\printbibliography



\end{document}