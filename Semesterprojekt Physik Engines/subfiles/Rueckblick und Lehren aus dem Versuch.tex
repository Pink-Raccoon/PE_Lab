\documentclass[../main.tex]{subfiles}

\begin{document}

    \newline
    Uns wurde beim Programmieren mit Unity ist es bewusst, wie wichtig es ist Gleitkommazahlen zu berücksichtigen,
    da Unity iterativ arbeitet. Dadurch entstanden Ungenauigkeiten, die das Experiment nicht korrekt durchlaufen
    liessen. Darüber hinaus war es erforderlich, bei den Labs verschiedene Aspekte wie Speicherung, Übergabe und
    Schleifen in der Programmierung zu berücksichtigen.
    \newline
    Besonders faszinierend ist die Art und Weise, wie Unity trotz seiner Funktion als Game-Engine mit den
    physikalischen Gesetzen umgeht. Obwohl es keine dedizierte Physik-Engine ist, erweist sich die Integration
    der physikalischen Aspekte als bemerkenswert.
    \newline
    Darüber hinaus hat es uns grossen Spass bereitet, uns sowohl mit der Physik als auch mit dem Programmieren
    intensiv auseinanderzusetzen. Die Kombination dieser beiden Bereiche war äusserst spannend und ermöglichte
    uns ein tieferes Verständnis der Zusammenhänge.
    \newline
    Insgesamt war dieses Projekt im Verlauf des Semesters herausfordernd, aber machbar. Die Idee mit den
    Bonuspunkten, d.h. eine maximale Note von 5.5, wenn Lab 2 abgeschlossen ist, und eine Note von 6.0,
    wenn alle 3 Labs erledigt wurden, finden wir sehr gut. Dies würde sich auch für zukünftige Studierende
    anbieten und motivierend wirken.


\end{document}