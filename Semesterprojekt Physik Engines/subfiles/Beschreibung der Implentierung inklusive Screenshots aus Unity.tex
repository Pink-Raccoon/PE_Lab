\documentclass[../main.tex]{subfiles}

\begin{document}
Im Code wurde der Würfel, welcher am anfangt beschleunigt wird, Romeo benannt und entsprechend der andere Würfel Julia. Die Würfeln werden in den nächsten Abschnitten ebenfalls so bezeichnet, damit man denn Zusammenhang mit den Screenshots vom Code leichter versteht. 
\\
\\
Für die Beschleunigung von Romeo wird, gemäss Berechnung, eine konstante Kraft von 4 (Newton) im FixedUpdate hinzugefügt. Es wurde auch bestummen, dass nach 2 Sekunden Romeo eine Geschwindigkeit von 2 m/s erreichen sollte und daraus eine ergibt sich eine Beschleunigungszeit von 1 Sekunde. Das Hinzufügen der Kraft passiert dementsprechend nur solange die Beschleunigungszeit noch nicht erreicht wurde.

Für den Teil mit dem elastischen Stoss an der Feder, wird die Auslenkung beim Start berechnet mit d


\end{document}