\documentclass[../main.tex]{subfiles}

\begin{document}
Im Code wurde der Würfel, welcher am anfangt beschleunigt wird, Romeo benannt und entsprechend der andere Würfel Julia. Die Würfeln werden in den nächsten Abschnitten ebenfalls so bezeichnet, damit man denn Zusammenhang mit den Screenshots vom Code leichter versteht. 
\\
\\
Für die Beschleunigung von Romeo wird, gemäss Berechnung, eine konstante Kraft von 4 (Newton) im FixedUpdate hinzugefügt. Es wurde auch bestummen, dass nach 2 Sekunden Romeo eine Geschwindigkeit von 2 m/s erreichen sollte und daraus eine ergibt sich eine Beschleunigungszeit von 1 Sekunde. Das Hinzufügen der Kraft passiert dementsprechend nur solange die Beschleunigungszeit noch nicht erreicht wurde.

Für den Teil mit dem elastischen Stoss, wurde ein Feder GameObject hinzugefügt. Die Werte für die Federauslenkung und Federkonstante wird beim Start ausgerechnet, bevor sich Romeo bewegt. Da beim zusammenstoss Romeo die Feder zusammendrückt, wird für die Auslenkung die Ruhelage der Feder genommen. Da in Unity sich die Position im Zentrum vom Objekt ist, 


\end{document}