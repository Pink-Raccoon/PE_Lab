\documentclass[../main.tex]{subfiles}

\begin{document}
    Während dem Durchlauf des Experimentes werden diverse Daten zu allen im Experiment vorkommenden
    Objekten gesammelt. Diese Daten umfassen Geschwindigkeit, Kinetische, sowie Potentielle Energie.
    Dabei wird zwischen dem Eleastischen sowie Inelastischen Zusammentoss unterschieden.
    \subsection{Elastisch}
    Nachfolgend werden alle Daten als Funktion der Zeit zu dem Elastischen Zusammenstoss in den Abbildungen
    Abbildung~\ref{fig:GeschwindigkeitAlsFunktionDerZeit} bis Abbildung~\ref{fig:PotentielleEnergieAlsFunktionDerZeit}
    aufgegliedert.

    \begin{figure}[H]
        \begin{center}
            \centerline{\includegraphics[width=155mm]{./images/Elastisch/GeschwindigkeitAlsFunktionDerZeit}}
            \caption{Geschwindigkeit als Funktion der Zeit}
            \label{fig:GeschwindigkeitAlsFunktionDerZeit}
        \end{center}
    \end{figure}

    \begin{figure}[H]
        \begin{center}
            \centerline{\includegraphics[width=155mm]{./images/Elastisch/KinetischeEnergieAlsFunktionDerZeit}}
            \caption{Kinetische Energie als Funktion der Zeit}
            \label{fig:KinetischeEnergieAlsFunktionDerZeit}
        \end{center}
    \end{figure}

    \begin{figure}[H]
        \begin{center}
            \centerline{\includegraphics[width=155mm]{./images/Elastisch/OrtAlsFunktionDerZeit}}
            \caption{Ort als Funktion der Zeit}
            \label{fig:OrtAlsFunktionDerZeit}
        \end{center}
    \end{figure}

    \begin{figure}[H]
        \begin{center}
            \centerline{\includegraphics[width=155mm]{./images/Elastisch/PotentielleEnergieAlsFunktionDerZeit}}
            \caption{Potentielle Energie als Funktion der Zeit}
            \label{fig:PotentielleEnergieAlsFunktionDerZeit}
        \end{center}
    \end{figure}

    \subsection{Inelastisch}
    Nachfolgend werden alle Daten als Funktion der Zeit zu dem Inelastischen Zusammenstoss in den Abbildungen
    Abbildung~\ref{fig:EnergieCube1AlsFunktionDerZeit} bis Abbildung~\ref{fig:OrtAlsFunktionDerZeitInelastisch}
    aufgegliedert.

    \begin{figure}[H]
        \begin{center}
            \centerline{\includegraphics[width=155mm]{./images/Inelastisch/EnergieCube1AlsFunktionDerZeit}}
            \caption{Energie Cube 1 als Funktion der Zeit}
            \label{fig:EnergieCube1AlsFunktionDerZeit}
        \end{center}
    \end{figure}

    \begin{figure}[H]
        \begin{center}
            \centerline{\includegraphics[width=155mm]{./images/Inelastisch/GeschwindigkeitCube1AlFunktionDerZeit}}
            \caption{Geschwindigkeit Cube 1 als Funktion der Zeit}
            \label{fig:GeschwindigkeitCube1AlFunktionDerZeit}
        \end{center}
    \end{figure}

    \begin{figure}[H]
        \begin{center}
            \centerline{\includegraphics[width=155mm]{./images/Inelastisch/GeschwindigkeitCube2AlsFunktionDerZeit}}
            \caption{Geschwindigkeit Cube 2 als Funktion der Zeit}
            \label{fig:GeschwindigkeitCube2AlsFunktionDerZeit}
        \end{center}
    \end{figure}

    \begin{figure}[H]
        \begin{center}
            \centerline{\includegraphics[width=155mm]{./images/Inelastisch/OrtAlsFunktionDerZeit}}
            \caption{Ort als Funktion der Zeit}
            \label{fig:OrtAlsFunktionDerZeitInelastisch}
        \end{center}
    \end{figure}


\end{document}