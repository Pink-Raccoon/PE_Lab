\documentclass[../main.tex]{subfiles}

\begin{document}
    \subsection{Grafiken zu Lab 2}
    Während dem Durchlauf des Experimentes des Lab 2, werden diverse Daten physikalische Vorgängen
    gesammelt. Diese Daten umfassen Ort und Geschwindigkeit, sowie kinetische und potentielle Energie.
    Dabei wird zwischen dem elastischen sowie inelastischen Zusammenstoss unterschieden.
    \subsubsection{Elastisch}
    Nachfolgend werden alle Daten als Funktion der Zeit zu dem elastischen Zusammenstoss in der
    Abbildung~\ref{fig:GeschwindigkeitAlsFunktionDerZeit} bis Abbildung~\ref{fig:OrtAlsFunktionDerZeit}
    aufgegliedert.

    \begin{figure}[H]
        \begin{center}
            \centerline{\includegraphics[width=155mm]{./images/Elastisch/GeschwindigkeitAlsFunktionDerZeit}}
            \caption{Geschwindigkeit als Funktion der Zeit}
            \label{fig:GeschwindigkeitAlsFunktionDerZeit}
        \end{center}
    \end{figure}

    \begin{figure}[H]
        \begin{center}
            \centerline{\includegraphics[width=155mm]{./images/Elastisch/OrtAlsFunktionDerZeit}}
            \caption{Ort als Funktion der Zeit}
            \label{fig:OrtAlsFunktionDerZeit}
        \end{center}
    \end{figure}

    \newpage
    \subsubsection{Inelastisch}
    Nachfolgend werden alle Daten als Funktion der Zeit zu dem inelastischen Zusammenstoss in der
    Abbildung~\ref{fig:gesamtImpluls} bis Abbildung~\ref{fig:Endgeschwindigkeit}
    aufgegliedert.
    \newline
    \newline
    In Abbildung~\ref{fig:gesamtImpuls} ist deutlich zu erkennen, wie Romeo
    während der Beschleunigungsphase in den ersten Sekunden an Impuls gewinnt. 
    Nach fünf Sekunden Gleitphase stösst Romeo auf die Feder, wodurch er abgebremst
    wird und Impuls verliert, bis er schliesslich bei null ankommt. Der Impuls wird
    jedoch in der gespannten Feder gespeichert und beim Entspannen der Feder wieder
    auf den Würfel übertragen. Dadurch hat der Würfel nach der Beschleunigungsphase 
    wieder den gleichen Impuls wie zuvor. Der Gesamtimpuls bleibt auch nach der Kollision
    mit Julia gleich, da das Diagramm den Impuls beider Würfel berücksichtigt.
    \begin{figure}[H]
        \begin{center}
            \centerline{\includegraphics[width=155mm]{./images/Inelastisch/GesamtImpluls}}
            \caption{GesamtImpluls als Funktion der Zeit}
            \label{fig:gesamtImpluls}
        \end{center}
    \end{figure}

    \begin{figure}[H]
        \begin{center}
            \centerline{\includegraphics[width=155mm]{./images/Inelastisch/ImpulsJulia}}
            \caption{Impuls Julia als Funktion der Zeit}
            \label{fig:ImpulsJulia}
        \end{center}
    \end{figure}

    \begin{figure}[H]
        \begin{center}
            \centerline{\includegraphics[width=155mm]{./images/Inelastisch/ImpulsRomeo}}
            \caption{Impuls Romeo als Funktion der Zeit}
            \label{fig:ImpulsRomeo}
        \end{center}
    \end{figure}

    \begin{figure}[H]
        \begin{center}
            \centerline{\includegraphics[width=155mm]{./images/Inelastisch/OrtJuliaAlsFunktionDerZeit}}
            \caption{Ort Julia als Funktion der Zeit}
            \label{fig:OrtJuliaAlsFunktionDerZeit}
        \end{center}
    \end{figure}

    \begin{figure}[H]
        \begin{center}
            \centerline{\includegraphics[width=155mm]{./images/Inelastisch/GeschwindigkeitRomeo}}
            \caption{Geschwindigkeit Romeo als Funktion der Zeit}
            \label{fig:GeschwindigkeitRomeo}
        \end{center}
    \end{figure}

    \begin{figure}[H]
        \begin{center}
            \centerline{\includegraphics[width=155mm]{./images/Inelastisch/GeschwindigkeitJulia}}
            \caption{Geschwindigkeit Julia als Funktion der Zeit}
            \label{fig:GeschwindigkeitJulia}
        \end{center}
    \end{figure}



    \begin{figure}[H]
        \begin{center}
            \centerline{\includegraphics[width=155mm]{./images/Inelastisch/Endgeschwindigkeit}}
            \caption{Endgeschwindigkeit als Funktion der Zeit}
            \label{fig:Endgeschwindigkeit}
        \end{center}
    \end{figure}



\end{document}