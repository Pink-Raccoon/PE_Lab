\documentclass[../main.tex]{subfiles}

\begin{document}
In diesem Kapitel werden die physikalische Vorgänge des Versuches beschrieben. Es geschehen drei Vorgänge, der Raketentrieb, einen elastischen Stoss und einen inelastischen Stoss.
Die gegebenen Massen sind:
\begin{itemize}
	\item Gewicht(m) = 2kg
	\item Velocity(v) = 2m/s
	\item Würfelseite = 1.5m
\end{itemize}

\subsection{Raketenantrieb\\}
Um die Kraft des Raketenantriebs zu berechnen nehmen wir die gewünschte Geschwindigkeit und berechnen damit die Beschleunigung,a. Da Kraft:\\ $F = m*a$.\\
Um dieses Anfangwertproblems zu lösen nehmen wir die Formel \\$\dot{v} = a$\\
$2m*s^{-1} \rightarrow$  $-2m*s^{-2} \rightarrow$  $a = [\frac{2m}{s^{2}}]$\\\\
Somit:
$F = 2kg * \frac{2m}{s^{2}} => \frac{4kg*m}{s^{2}} = 4N$\\
4N werden deshalb als konstante Kraft angewendet, damit auch die gewünschte Geschwindigkeit erreicht wird.








\subsection{Elastischer Stoss}
Beim elastischen Stoss ist die kinetische Energie vom Stosspartner vor und nach der Kollision gleich.
Kinetische Energie wird mit folgender Formel berechnet:\\
$\frac{1}{2} * m * v^{2}$ \\
Setzt man die Massen von diesem Projekt ein bekommt man:\\

$\frac{1}{2} * 2kg * (\frac{2m}{s})^{2} = 4J$

\subsection{Inelastischer Stoss}




\subsection{}

\end{document}