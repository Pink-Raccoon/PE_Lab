\documentclass{article}[11pt,paper=A4, margin=1cm]
% Language setting
% Replace `english' with e.g. `spanish' to change the document language
\usepackage[german]{babel}

% Set page size and margins
% Replace `letterpaper' with `a4paper' for UK/EU standard size
\usepackage[letterpaper,top=2cm,bottom=2cm,left=3cm,right=3cm,marginparwidth=1.75cm]{geometry}% Useful packages
\usepackage{amsmath}
\usepackage{amsfonts} 
\usepackage{amssymb}
\usepackage{mathtools}
\usepackage{graphicx}
\usepackage[colorlinks=true, allcolors=blue]{hyperref}
\usepackage{float}
\usepackage{multicol}
\usepackage{xcolor}
\usepackage{amsfonts}
\usepackage[utf8]{inputenc}
\usepackage{pgfplots}
\usepackage{listings}
\usepackage[linewidth=1pt]{mdframed}
\usepackage{lipsum}
\usepackage{subfiles} % Best loaded last in the preamble
\usepackage[backend=biber]{biblatex}
\usepackage{csquotes}
%=================================
%Code Formatting
\usepackage{listings}
\begin{document}
\section{Abstract}
\title{Retrotool}
Sprint Retrospektiven bewähren sich innerhalb der agilen Arbeitsweise mit SCRUM. Sie sind Katalysatoren für Veränderungen, wobei ihr Fokus sowohl auf Management- als auch auf Entwicklungsprozessen liegt. Sie erhöhen die Effizienz und Effektivität der Team-Bemühungen, indem, durch regelmässige Besprechungen der aktuellen Erfahrungen jedes einzelnen Teammitglieds, Entwicklungsfehler frühzeitig diskutiert, eigene Stärken und Schwächen analysiert und durch den Wissenstransfer der Mitglieder behoben werden können. Nicht nur das vorliegend berichtende Entwicklungsteam stellt in seiner Arbeitsumgebung fest, dass das Remote-Arbeiten in der Industrie an Bedeutung gewinnt. Das vom Team entwickelte «Retrotool» trägt ebendiesem Umstand Rechnung. Das Retrotool ist als Browser-Applikation konzipiert und kann zudem über Mobiltelefone bedient werden. Das Backend ist mit C, das Frontend mit TypeScript und einem Ionic-Framework implementiert, basierend auf React. Diese Lösung ist bewusst einfach gehalten, beispielsweise durch den Verzicht auf eine Datenbank, um die Benutzerfreundlichkeit und Einrichtung zu erleichtern. Während der Entwicklung wurde das Tool bereits vom Entwicklungsteam für ihre eigenen Retrospektiven genutzt und hat sich sowohl in der Bedienung als auch in der Remote-Arbeit bewährt. Sie erleichtert die Zusammenarbeit auf Distanz und erhöht die Effektivität der in diesem Rahmen durchgeführten Retrospektiven, wie die Teammitglieder bestätigen. Allerdings ist das Ionic-Framework aufgrund unzureichender Dokumentation nicht optimal. Ob es eine Weiterentwicklung von «Retrotool» geben wird, hängt von aktuell noch offenen Fragen ab, wie z.B. dem Hosting der Browser-Applikation und der Möglichkeit einer Cloud-Lösung. Zudem muss das Entwicklungsteam das Kostenmodell klären, bevor es mit einer Weiterentwicklung voranschreitet. Wenn sich Teammitglieder oder andere Interessierte für eine marktorientierte Weiterentwicklung oder private Nutzung des Retrotools interessieren, sind Fragen des geistigen Eigentums zu berücksichtigen.

\newline
\newline
Keywords: Agile Arbeitsweise, Sprint Retrospektive, Teamarbeit, Remote-Arbeit, Browser-Applikation, «Retrotool». 
Namen der Entwickler: Marius Blöchlinger, Philip Kuttelwascher, Paul Müller, Asha Schwegler, Michel Steiner, Kim Lan Vu 

Datum: 28.5.2023
\end{document}
