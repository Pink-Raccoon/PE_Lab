\documentclass[../main.tex]{subfiles}
% Packages

%=================================
%Code Formatting




% Document
\begin{document}


    \subsection{Code für das Experiment}
    Nachfolgend ist der für das Experiment benötigte Code abgebildet.
    Dieser Umfasst alle Physikalischen berechnungen, sowie die Bewegungen.

%    \lstinputlisting[caption=Scheduler, style=customc]{..//UnityProj/Assets/CubeController.cs}
    \begin{lstinputlisting}[label={lst:CubeController}]
        {..//UnityProj/Assets/CubeController.cs}
    \end{lstinputlisting}

    \subsection{Code für die Datenaufbereitung des Elastischen Stosses}
    Nachfolgend ist der Code abgebildet, welche für die Visuelle Datenaufbereitung des Elastischen Stosses
    der Grafiken benötigt wurde.
    \begin{lstinputlisting}[label={lst:Elastischen}]
    {..//UnityProj/graph.py}
    \end{lstinputlisting}

    \subsection{Code für die Datenaufbereitung des Inelastischen Stosses}
    Nachfolgend ist der Code abgebildet, welche für die Visuelle Datenaufbereitung des Inelastischen Stosses
    der Grafiken benötigt wurde.
    \begin{lstinputlisting}[label={lst:graphInelastic}]
        {..//UnityProj/graphInelastic.py}
    \end{lstinputlisting}

\end{document}