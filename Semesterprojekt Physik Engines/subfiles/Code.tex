\documentclass[../main.tex]{subfiles}
% Packages

%=================================
%Code Formatting




% Document
\begin{document}


    \subsection{Code für das Lab 2 und 3}
    Nachfolgend sind die einzelnen Codesfile aufgelistet, die für den gesamten Labaufbau benötigt wurden.

    \lstinputlisting[caption=\textbf{CubeController.cs}, style=customc]{\cubeControllerFile}
    \begin{lstinputlisting}[label={lst:CubeController}]
        {\cubeControllerFile}
    \end{lstinputlisting}

    \lstinputlisting[caption=\textbf{LineJulia.cs}, style=customc]{\lineJuliaFile}
    \begin{lstinputlisting}[label={lst:CubeController}]
    {\lineJuliaFile}
    \end{lstinputlisting}


    \lstinputlisting[caption=\textbf{RopeSpawn.cs}, style=customc]{\ropeSpawnFile}
    \begin{lstinputlisting}[label={lst:CubeController}]
    {\ropeSpawnFile}
    \end{lstinputlisting}


    \lstinputlisting[caption=  \textbf{SwingJulia.cs}, style=customc]{\swingJuliaFile}
    \begin{lstinputlisting}[label={lst:CubeController}]
    {\swingJuliaFile}
    \end{lstinputlisting}


    \lstinputlisting[caption=\textbf{RopeSpawn.cs}, style=customc]{\ropeSpawnFile}
    \begin{lstinputlisting}[label={lst:CubeController}]
    {\ropeSpawnFile}
    \end{lstinputlisting}


    \lstinputlisting[caption=\textbf{SwingRomeo.cs}, style=customc]{\swingRomeoFile}
    \begin{lstinputlisting}[label={lst:CubeController}]
    {\swingRomeoFile}
    \end{lstinputlisting}


    \subsection{Code für die Datenaufbereitung des Elastischen Stosses des Lab 2}
    Nachfolgend ist der Code abgebildet, welche für die Visuelle Datenaufbereitung des Elastischen Stosses
    der Grafiken benötigt wurde.
    \begin{lstinputlisting}[label={lst:Elastischen}]
    {..//UnityProj/graphElastic.py}
    \end{lstinputlisting}

    \subsection{Code für die Datenaufbereitung des Inelastischen Stosses des Lab 2}
    Nachfolgend ist der Code abgebildet, welche für die Visuelle Datenaufbereitung des Inelastischen Stosses
    der Grafiken benötigt wurde.
    \begin{lstinputlisting}[label={lst:graphInelastic}]
        {..//UnityProj/graphInelastic.py}
    \end{lstinputlisting}

    \subsection{Code für die Datenaufbereitung des Schwunges von Julia des Lab 3}
    Nachfolgend ist der Code abgebildet, welche für die Visuelle Datenaufbereitung des Schwung von Julia.
    \begin{lstinputlisting}[label={lst:graphSwingJulia}]
    {..//UnityProj/graphSwingJulia.py}
    \end{lstinputlisting}

    \subsection{Code für die Datenaufbereitung des Schwunges von Romeo des Lab 3}
    Nachfolgend ist der Code abgebildet, welche für die Visuelle Datenaufbereitung des Schwung von Romeo.
    \begin{lstinputlisting}[label={lst:graphSwingRomeo}]
    {..//UnityProj/graphSwingRomeo.py}
    \end{lstinputlisting}

\end{document}