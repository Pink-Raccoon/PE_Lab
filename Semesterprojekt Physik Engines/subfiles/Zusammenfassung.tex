\documentclass[../main.tex]{subfiles}

\begin{document}
An der ZHAW wird im Physik Engine, kurz PE, werden physikalische Zusammenhänge
von Bewegungsrichtungen und Beschleunigung, sowie potenzieller und kinetischer Energie im Raum und Zeit
durch Simulationen auf Körper untersucht. Dabei werden über das Semester in Kleingruppen
zwei Labs bearbeitet.
\newline
\newline
Im ersten Lab soll ein Würfel mit einer konstanter Kraft beschleunigt werden und elastisch an einer
Wand abprallen ehe er wieder in die Richtung aus der er gekommen ist zurück gleitet. Nach dem elastischen
Stoss soll er mit einem zweiten Würfel inelastisch zusammen prallen, sodass beide gemeinsam weiter gleiten.
\newline
\newline
Im zweiten Lab sollen die zusammengeprallten Würfel mit Hilfe eines Kranes von der Gleitbewegung in eine
Pendelbewegung übergehen. Durch den Luftwiderstand wird somit Energie entzogen, ehe die Würfel zum
Stillstand kommen.
\newline
\newline
Diese Untersuchen werden werden mit den physikalischen Gesetzen in der Game-Engine Unity simuliert.
Dabei werden laufend daten zu den aktuellen Parameter einzelner Objekte gesammelt und mit diversen
Grafiken veranschaulicht. Die aus den zwei Experimenten gewonnenen Erkenntnisse werden in diesem Bericht
niedergeschrieben.



\end{document}